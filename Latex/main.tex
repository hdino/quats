\documentclass[11pt,a4paper]{scrartcl}
\usepackage[utf8]{inputenc}
\usepackage{amsmath}
\usepackage{amsfonts}
\usepackage{amssymb}
\usepackage{graphicx}
\usepackage{bm}
\usepackage{units}

\let\arrowedvec=\vec
\renewcommand\vec{\bm}
\newcommand{\quat}[1]{\mathbf{#1}}
\newcommand{\Matrix}[1]{\mathbf{#1}}
\newcommand{\skewop}[1]{\left[ #1 \right]_\times}
\newcommand{\qprod}{\otimes}

\author{Dino Hüllmann}
\title{Kalman, Quaternionen & Co.}

\begin{document}

\section{Einleitung}

Quaternion:
\begin{equation}
	Q = q_w + q_x i + q_y j + q_z k \in \mathbb{H}
\end{equation}

As a vector:
\begin{equation}
	\quat{q}
	=
	\begin{bmatrix}
		q_w \\ \vec{q}_v
	\end{bmatrix}
	=
	\begin{bmatrix}
		q_w \\ q_x \\ q_y \\ q_z
	\end{bmatrix}
\end{equation}

\subsection{Product}

\begin{equation}
	\quat{p} \qprod \quat{q}
	=
	\begin{bmatrix}
		p_w q_w - \vec{p}_v^\intercal \vec{q}_v \\
		p_w \vec{q}_v + q_w \vec{p}_v + \vec{p}_v \times \vec{q}_v
	\end{bmatrix}
\end{equation}

\begin{equation}
	\quat{q}_1 \qprod \quat{q}_2 = \left[\quat{q}_1\right]_L \quat{q}_2
	\qquad \text{and} \qquad
	\quat{q}_1 \qprod \quat{q}_2 = \left[\quat{q}_2\right]_R \quat{q}_1
\end{equation}

\begin{equation}
	\left[\quat{q}\right]_L
	=
	q_w \Matrix{I}
	+
	\begin{bmatrix}
		0 && -\quat{q}_v^\intercal \\
		\quat{q}_v && \skewop{\quat{q}_v}
	\end{bmatrix}\text{,}	
	\qquad
	\left[\quat{q}\right]_R
	=
	q_w \Matrix{I}
	+
	\begin{bmatrix}
	0 && -\quat{q}_v^\intercal \\
	\quat{q}_v && -\skewop{\quat{q}_v}
	\end{bmatrix}
\end{equation}

\section{Rotation}

\begin{equation}
	\Matrix{R}(t)
	=
	\Matrix{R}(0) \exp \left( \skewop{\vec{\omega}} t \right)
	=
	\Matrix{R}(0) \exp \left( \skewop{\vec{\omega} t} \right)
\end{equation}

Given the rotation vector $\vec{u} = \phi \vec{u}$, denoting a right-handed rotation of $\unit[\phi]{rad}$ around the unit axis given by $\vec{u}$:

\begin{equation}
	\quat{q}\lbrace\vec{v}\rbrace
	=
	\exp(\vec{v}/2)
	=
	\begin{bmatrix}
		\cos(\phi/2) \\ \vec{u} \sin(\phi/2)
	\end{bmatrix}
\end{equation}

\begin{equation}
	\Matrix{R}\lbrace\vec{v}\rbrace
	=
	\exp \left( \skewop{\vec{v}} \right)
	=
	\Matrix{I} + \sin \phi \skewop{\vec{u}} + (1 - \cos \phi) \skewop{\vec{u}}^2
\end{equation}

\section{Hamilton convention}

Order:
\begin{math}
	\left( q_w \text{, } \vec{q}_v \right)
\end{math}

Right-to-left products mean: Local-to-Global, i.e.
\begin{math}
	\vec{x}_\mathcal{G} = \quat{q} \qprod \vec{x}_\mathcal{L} \qprod \quat{q}^*
\end{math}

\section{Notation}

\subsection{Translation}

\begin{equation}
	\vec{t}_{[\text{expressed in}], [\text{initial point}][\text{end point}]}
\end{equation}

Example: $\vec{t}_{A,BC}$

Point: $\vec{p}_A = \vec{t}_{A,AP}$

Vector $\vec{v} = \overline{BC}$: $\vec{v}_A = \vec{t}_{A,\overline{BC}}$

In case of the world frame $\mathcal{W}$: $\vec{p} = \vec{p}_\mathcal{W} = \vec{t}_{\mathcal{W}, \mathcal{W} P}$

\subsection{Rotation}

a transformation \textit{from} frame $B$ \textit{to} frame $A$ or a transformation \textit{of} frame $B$ \textit{with respect to} frame $A$

\begin{equation}
	\quat{q}_{[\text{to}][\text{from}]}
	\equiv
	\quat{q}_{[\text{with respect to}][\text{of}]},
	\qquad
	\Matrix{R}_{[\text{to}][\text{from}]}
	\equiv
	\Matrix{R}_{[\text{with respect to}][\text{of}]}
\end{equation}

\section{Integration}

\subsection{Zeroth order backward integration}

\subsection{First order integration}

\section{Jacobians of the rotation}

Rotation of vector $\vec{a}$ of $\theta$ radians around unit axis $\vec{u}$.

\begin{equation}
	\vec{\theta} = \theta \vec{u},
	\qquad
	\quat{q} = \quat{q}\lbrace \vec{\theta} \rbrace,
	\qquad
	\Matrix{R} = \Matrix{R}\lbrace \vec{\theta} \rbrace
\end{equation}

\subsection{Jacobian with respect to the vector}

\begin{equation}
	\frac{\partial(\Matrix{R} \vec{a})}{\partial \vec{a}} = \Matrix{R}
\end{equation}

\subsection{Jacobian with respect to the rotation vector}

\begin{equation}
	\frac{\partial(\Matrix{R} \vec{a})}{\partial \vec{\theta}}
	=
	-\skewop{\Matrix{R} \vec{a}}
\end{equation}

\subsection{Jacobian with respect to the quaternion}

Define: $\quat{q} = \begin{bmatrix}w && \vec{v}\end{bmatrix}$

\begin{equation}
	\frac{\partial(\Matrix{R} \vec{a})}{\partial \quat{q}}
	=
	2 \begin{bmatrix}
		w \vec{a} + \vec{v} \times \vec{a} &&
		\vec{v}^\intercal \vec{a} \Matrix{I} + \vec{v} \vec{a}^\intercal - \vec{a} \vec{v}^\intercal - w \skewop{\vec{a}}
	\end{bmatrix}
\end{equation}

\section{Error-state Kalman filter (ESKF)}

True-, nominal- and error-state values. The true-state is expressed as a composition of the nominal- and the error-states.

Nominal-state: $\vec{x}$

Error-state: $\delta \vec{x}$

\begin{enumerate}
	\item High-frequency data is integrated into the nominal-state, in parallel the ESKF predicts a Gaussian estimate of the error-state.
	\item Low-frequency measurements arrive that render the errors used for filter correction.
	\item The posterior Gaussian estimate of the error-state is computed.
	\item The error-state's mean is injected into the nominal-state and then reset to zero.
	\item The error-state's covariance matrix is updated to reflect this reset.
\end{enumerate}

\subsection{The true-state dynamics}

\begin{subequations}
\begin{eqnarray}
	\dot{\quat{q}}_t = \frac{1}{2} \quat{q}_t \qprod \vec{\omega}_t \\
	\dot{\vec{\omega}}_{bt} = \vec{\omega}_w
\end{eqnarray}
\end{subequations}

\begin{tabular}{l c c c c}
	Magnitude & True & Nominal & Error & Composition \\ \hline
	Quaternion & $\quat{q}_t$ & $\quat{q}$ & $\delta \quat{q}$ & $\quat{q}_t = \quat{q} \qprod \delta \quat{q}$ \\
	Gyroscope bias & $\vec{\omega}_{bt}$ & $\vec{\omega}_b$ & $\delta \vec{\omega}_b$ & $\vec{\omega}_{bt} = \vec{\omega}_b + \delta \vec{\omega}_b$
\end{tabular}

\subsection{Derivation}

True-state:
\begin{equation}
	\vec{x}_t
	=
	\begin{bmatrix}
		\quat{q}_t \\ \vec{\omega}_{bt}
	\end{bmatrix}
\end{equation}

System model:
\begin{equation}
	\dot{\quat{q}}_t = \frac{1}{2} \quat{q}_t \qprod \vec{\omega}_t \medspace,
	\qquad
	\dot{\vec{\omega}}_{bt} = \vec{\omega}_w
\end{equation}

Measurements:
\begin{subequations}
\begin{eqnarray}
	\vec{a}_m = \Matrix{R}_t^\intercal \left( \vec{a}_t - \vec{g} \right) + \vec{a}_n \\
	\vec{\omega}_m = \vec{\omega}_t + \vec{\omega}_{bt} + \vec{\omega}_n
\end{eqnarray}
\end{subequations}
with $\Matrix{R}_t := \Matrix{R}(\quat{q}_t)$.

Error state:
\begin{equation}
	\delta \vec{x}
	=
	\begin{bmatrix}
		\delta \vec{\theta} \\ \delta \vec{\omega}_b
	\end{bmatrix}
\end{equation}

\minisec{Measurement Jacobian}

Sensor model:
\begin{equation}
	\vec{y} = h(\vec{x}_t) + \vec{v}
\end{equation}

We had:
\begin{equation}
	\vec{a}_m = \Matrix{R}_t^\intercal \left( \vec{a}_t - \vec{g} \right) + \vec{a}_n
\end{equation}

Assumption: the mean of $\vec{a}_t$ is zero. Therefore,
\begin{equation}
	\underbrace{\vec{a}_m}_{\vec{y}}  = \underbrace{\Matrix{R}_t^\intercal \left( -\vec{g} \right)}_{h(\vec{x}_t)} + \underbrace{\vec{a}_n}_{\vec{v}}
\end{equation}

\begin{equation}
	h(\vec{x}_t)
	=
	\Matrix{R}_t^\intercal \left( -\vec{g} \right)
	=
	\Matrix{R}\left( \quat{q}_t^* \right) \left( -\vec{g} \right)
\end{equation}

\begin{equation}
	\Matrix{H}
	=
	\left. \frac{\partial h}{\partial \delta \vec{x}} \right|_{\vec{x}}
	=
	\left. \frac{\partial h}{\partial \vec{x}_t} \right|_{\vec{x}} \left. \cdot \frac{\partial \vec{x}_t}{\partial \delta \vec{x}} \right|_{\vec{x}}
	=
	\Matrix{H}_x \cdot \Matrix{H}_{\delta x}
\end{equation}

\begin{equation}
	\Matrix{H}_{\delta x}
	=
	\begin{bmatrix}
		\left.\frac{\partial \left( \quat{q} \qprod \delta \quat{q} \right)}{\partial \delta \vec{\theta}}\right|_{\vec{q}} & \Matrix{0} \\
		\Matrix{0} & \left.\frac{\partial \left( \vec{\omega}_b + \delta \vec{\omega}_b \right)}{\partial \delta \vec{\omega}_b}\right|_{\vec{\omega}_b}
	\end{bmatrix}
\end{equation}

\begin{equation}
	\left.\frac{\partial \left( \vec{\omega}_b + \delta \vec{\omega}_b \right)}{\partial \delta \vec{\omega}_b}\right|_{\vec{\omega}_b}
	=
	\Matrix{I}_3
\end{equation}

\begin{equation}
	\left.\frac{\partial \left( \quat{q} \qprod \delta \quat{q} \right)}{\partial \delta \vec{\theta}}\right|_{\vec{q}}
	=
	\left[\quat{q}\right]_L \frac{1}{2}
	\begin{bmatrix}
	\Matrix{0}_{1 \times 3} \\ \Matrix{I}_3
	\end{bmatrix}
\end{equation}

\begin{equation}
	\Matrix{H}_x = \left. \frac{\partial h}{\partial \vec{x}_t} \right|_{\vec{x}}
	=
	\begin{bmatrix}
		\left.\frac{\partial h}{\partial \quat{q}_t}\right|_{\quat{q}} &
		\left.\frac{\partial h}{\partial \vec{\omega}_{bt}}\right|_{\vec{\omega}_b}
	\end{bmatrix}
\end{equation}



\end{document}